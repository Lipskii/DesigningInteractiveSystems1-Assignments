\section{Solution \#3}

\subsection{Objective of the study}
To find whether our prototype about Shopping Healthier would be understandable by users and whether users get fun while using this system. Besides, we observe if the users have problems with the usage.

\subsection{Contextual information}
For that is complex to perform the tasks with our low-fidelity prototype in a real supermarket(a supermarket is crowded and we cannot face to emergency scenarios with our low-fi prototype), we have chosen the building of the RWTH Computer Science Falculty to ask our interviewees to perform the tasks.

\subsection{Details of the task}
We use the Model Extraction to see if user have problems with the usage and whether user can  perceive the information from the system well. We ask our interviewees hold our prototype in hand and perform normal shopping process. We show the interaction while the users are shopping without explaning and the user should speak out their thinking of the information they perceived from the prototype. The process takes about 5 to 7 minutes.

\subsection{Experimental procedure}


\subsection{Unedited video}
Please see the folder "shopping healthier".

%\subsection{Notes collected during evaluation(optional)}

\subsection{Important findings}
\subsubsection{User A: Tina}




\subsubsection{User B: Enes}
