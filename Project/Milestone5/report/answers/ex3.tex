\section{Solution \#3}

\subsection{Objective of the study}
To find whether our prototype about Shopping Healthier would be understandable by users and whether users get fun while using this system. Besides, we observe if the users have problems with the usage.

\subsection{Contextual information}
For that is complex to perform the tasks with our low-fidelity prototype in a real supermarket(a supermarket is crowded and we cannot face to emergency scenarios with our low-fi prototype), we have chosen the building of the RWTH Computer Science Falculty to ask our interviewees to perform the tasks.

\subsection{Details of the task}
We use the Model Extraction to see if user have problems with the usage and whether user can  perceive the information from the system well. We ask our interviewees hold our prototype in hand and perform normal shopping process. We show the interaction while the users are shopping without explaning and the user should speak out their thinking of the information they perceived from the prototype. The process takes about 5 to 7 minutes.

\subsection{Experimental procedure}
%Each of the user will follow the same path of the experiment. Before evaluation users will have no idea about out solutions. Additionally we will perform Retrospective Testing.
\begin{itemize}
  \item Initial enviroment setup: we placed the items from the supermarket on the desk to emultate the supermarket enviroment.
  \item Ask the user to do normal shopping process(as in a supermarket).
  \item While the user is shopping items, show the prototype's interaction to the user without explanation.
  \item Ask the user to inform experimenters when the user is done with the whole shopping process to compelete this experimental procedure.
Additionaly, we perform Retrospective Testing.
\end{itemize}


\subsection{Unedited video}
Please see the folder "shopping healthier".
%\subsection{Notes collected during evaluation(optional)}

\subsection{Important findings}
\begin{itemize}
 \item For both users it was unclear at the begining how the redeem procedure and  the "feeding the sheep" system
works. One user suggested to use some kind of pop up message with explanation of the system works.
 \item One of the users liked the idea of choosing her diet, however she noted that more types of diet should be included
 \item Both of the users did enjoy the redeem points system and therefore found this solution fun.
 \item One of the user issue was that he didn't know what exactly to do after he finished shopping. He suggested that there should be a description saying, that user should scan the bardcode afterwards at the check-out.
 \end{itemize}
