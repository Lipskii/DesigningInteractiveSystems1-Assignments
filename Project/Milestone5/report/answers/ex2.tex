\section{Solution \#2}

\subsection{Objective of the study}
To find whether our prototype about Shopping Companion would be well-received or not



\subsection{Contextual Information}
the experiment held in a hall in Computer Engineering Department of RWTH. There were passerby people, which we cannot avoid. This might have created tension on the user, yet it was our best call.


\subsection{Details of the task}
The Model Extraction is employed to extract what user understand from the prototype. In addition, this prototype employs a genie, who can talk and direct the user. In order to emulate that with our low budget, I played the role of the genie. (I only read the text on the interface, and did not add any other word in my speech to the users)

\subsection{Experimental Procedure}

\begin{itemize}
	\item User is greeted, and her consent is openly asked.
	\item The items on the left (items from a supermarket) are explained, that they are actual items from supermarket and they are emulating supermarket aisles.
	\item The prototype is explained briefly, that the paper belongs to the application UI and how one can interact with that.
	\item Pre-provided cellphone with a shopping list on it is explained to the user.
	\item The user synchronizes using the given phone.
	\item The shopping list appears on the tablet, and the genie encourages user to start scanning items.
	\item The user scans the shopping items on her left in any order she wants.
		\begin{itemize}
			\item whenever an item is scanned, an appropriate UI change (information about the scanned item, crossing out the item from the shopping list, advice on health) has been made.
		\end{itemize}
	\item as the shopping list is depleted, the user is informed about this event and asked whether she wants to continue shopping. In case the user wants to continue, UI goes back to displaying shopping list and is ready to start upcoming items, if there is any.
\end{itemize}

\subsection{Unedited video}
Please see the following YouTube links
\begin{itemize}
	\item Tina
		\begin{itemize}
			\item UI and user interaction videos, merged side-by-side into one: \url{videos/shopping companion/tina_test.mp4}
			
			\item the user feedback after the experiment: \url{videos/shopping companion/tina_feedback.mp4}
		\end{itemize}
	
	\item Enes
	\begin{itemize}
		\item UI and user interaction videos, merged side-by-side into one: \url{videos/shopping companion/enes_test.mp4}
		
		\item the user feedback after the experiment: \url{videos/shopping companion/enes_feedback.mp4}
	\end{itemize}
\end{itemize}

\subsection{Important findings}

	\begin{itemize}
		\item Users liked the voice feedback that is provided. (we were not sure about whether to provide voice-over or not, now it is evident that it is beneficial) (was also a tip from our advisor)
		\item the writing style of the comment on unhealthy products are described to be a bit intrusive: it was also said we might improve on the word selection to make user feel at ease \& not feel to be judged.
		\item using different screens for the comment on some of the products misled the user, and required more time from them to interpret. It is better to stick with single same screen for item identification, and provide the comment there, not in a different page.
		\item the page about the shopping list completion rate seemed a bit different, and user tended to continue shopping. In order to aid this, we decided to remove this page altogether and only provide an exclamation mark over the completion percentage in GUI to make the experience more enjoyable.
		\item feedback on the health concern over some items seemed like a bit off by the users. A recommendation on allowing a user to enable/disable the food health comments is made, which is well-received. On the initial screen, we would provide a switch for user to decide on whether she wants to enable comments on food quality.
	\end{itemize}

